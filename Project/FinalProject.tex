\DocumentMetadata{pdfversion=2.0}
\documentclass[conference]{IEEEtran}
% \IEEEoverridecommandlockouts
% The preceding line is only needed to identify funding in the first footnote. If that is unneeded, please comment it out.
\usepackage{cite}
\usepackage{amsmath,amssymb,amsfonts}
\usepackage{algorithmic}
\usepackage{graphicx}
\usepackage{textcomp}
\usepackage{xcolor}
\usepackage[colorlinks=true,urlcolor=blue,linkcolor=blue]{hyperref} % To allow Alt Text and Linking for Headers, and to allow blue, underlined URL hyperlinks
\usepackage{natbib} % To allow BibTeX citations

\def\BibTeX{{\rm B\kern-.05em{\sc i\kern-.025em b}\kern-.08em
    T\kern-.1667em\lower.7ex\hbox{E}\kern-.125emX}}
\begin{document}

\title{Systematic Literature Review on the Semantic Analysis of Cybersecurity Indicators
% {\footnotesize \textsuperscript{*}Note: Sub-titles are not captured in Xplore and
% should not be used}
% \thanks{Identify applicable funding agency here. If none, delete this.}
}

\author{\IEEEauthorblockN{Nic Recasens}
\IEEEauthorblockA{\textit{Computer Science} \\
\textit{Georgia Southern University}\\
Statesboro, Georgia, USA \\
jr38088@georgiasouthern.edu}
\and
\IEEEauthorblockN{Nigel Smith}
\IEEEauthorblockA{\textit{Computer Science} \\
\textit{Georgia Southern University}\\
Statesboro, Georgia, USA \\
ns15468@georgiasouthern.edu}}

\maketitle

%TODO @Nic Do you have any issues with making the sections non-numbered? I prefer that for readability, but am not married to it.
\section*{Introduction}
Introduce the project topic, its importance, and the scope of the review.

\section*{Methodology}
Explain the process of selecting articles, detailing:
\begin{itemize}
    \item Keywords used
    \item Databases accessed
    \item Inclusion and exclusion criteria
\end{itemize}

\section*{Literature Review}
Summarize and critically analyze each article, focusing on:

\section*{Synthesis and Conclusion}
Highlight overall trends and research gaps, and describe how your project addresses these gaps.

% \subsection{Identify the Headings}
% Headings, or heads, are organizational devices that guide the reader through 
% your paper. There are two types: component heads and text heads.

% Component heads identify the different components of your paper and are not 
% topically subordinate to each other. Examples include Acknowledgments and 
% References and, for these, the correct style to use is ``Heading 5''. Use 
% ``figure caption'' for your Figure captions, and ``table head'' for your 
% table title. Run-in heads, such as ``Abstract'', will require you to apply a 
% style (in this case, italic) in addition to the style provided by the drop 
% down menu to differentiate the head from the text.

% Text heads organize the topics on a relational, hierarchical basis. For 
% example, the paper title is the primary text head because all subsequent 
% material relates and elaborates on this one topic. If there are two or more 
% sub-topics, the next level head (uppercase Roman numerals) should be used 
% and, conversely, if there are not at least two sub-topics, then no subheads 
% should be introduced.

\bibliographystyle{apalike} 
\bibliography{myrefs}

\end{document}