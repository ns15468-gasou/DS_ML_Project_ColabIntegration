\DocumentMetadata{pdfversion=2.0}
\documentclass[conference]{IEEEtran}
% \IEEEoverridecommandlockouts
% The preceding line is only needed to identify funding in the first footnote. If that is unneeded, please comment it out.
\usepackage{cite}
\usepackage{amsmath,amssymb,amsfonts}
\usepackage{algorithmic}
\usepackage{graphicx}
\usepackage{textcomp}
\usepackage{xcolor}
\usepackage[colorlinks=true,urlcolor=blue,linkcolor=blue]{hyperref} % To allow Alt Text and Linking for Headers, and to allow blue, underlined URL hyperlinks

\def\BibTeX{{\rm B\kern-.05em{\sc i\kern-.025em b}\kern-.08em
    T\kern-.1667em\lower.7ex\hbox{E}\kern-.125emX}} % Previous lines from IEEE template
\begin{document}

\title{Systematic Literature Review on the Semantic Analysis of Cybersecurity Indicators}

\author{\IEEEauthorblockN{Nic Recasens}
\IEEEauthorblockA{\textit{Computer Science} \\
\textit{Georgia Southern University}\\
Statesboro, Georgia, USA \\
jr38088@georgiasouthern.edu}
\and
\IEEEauthorblockN{Nigel Smith}
\IEEEauthorblockA{\textit{Computer Science} \\
\textit{Georgia Southern University}\\
Statesboro, Georgia, USA \\
ns15468@georgiasouthern.edu}}

\maketitle

\section{Introduction} % @j-recasens Unless you already had an Intro and Methodology stewing, here are some rough ones I'll clean up
% I'll leave the conclusion to you, and also very much feel free to change up the intro to more align with what you are thinking for
% our project idea, since I leave the overall vision more in your hands.
%%Introduce the project topic, its importance, and the scope of the review.
In the modern world, attackers are a step ahead of all Cybersecurity professionals. 
As the popular adage goes "Defenders have to succeed every time, but Attackers only have to succeed once" TODO cite this.
As such, the realm of machine learning and predictive analytics is of known importance in Cybersecurity research.
For decades, machine learning analysis has helped keep defenders on top of their game by preventing novel attacks from gaining footholds.
In the 90s, when the number of attackers was smaller, professionals were able to get away with only blocking known threats.
But with Ransomware and Malware becoming billion dollar multi-national black markets, we need to be able to apply risk trends to novel data.
At the forefront of this prevention is the identification and blocking of spam and phishing.
An individual datum related to spam and phishing is called an indicator of attack (IoA), and they let us know if someone is targeted by an attacker.
And if they are used successfully, they become indicators of compromise (IoC).
IoCs have the negative reputation of being indicative of failed Cybersecurity protection, but in relevance to this project, they are the second half of the puzzle that is semantic analysis of indicators.
Via the correlation of IoAs and/or IoCs, numerous researchers have pushed the boundary forward over the decades since the aforementioned 90s.
The attackers may be numerous in their successes, but they leave behind their own value aswell: data. 
With that data, we can see how pervasive cyber threats are, train models to simulate said attackers, and allow the very computers they attack help mitigate the risks of our modern interconnected world. 
Through this literature review, we will go over 8 papers that show how researchers achieved these goals, and show through their research how important cyber-risk mitigation truly is.

\section{Methodology}
%% Explain the process of selecting articles, detailing:
%Keywords used
At the most basic level, the 3 keywords we used to guide our literature selection process were "spam", "phishing", and "semantic".
Spam and Phishing are the strongest indicators Cybersecurity professionals have as indicators of attack. In fact, 
TODO - Source showing how Business Email Compromise (BEC), is known as the largest threat-.
As such, once we have papers scoped to the biggest risk facing the largest number of end users, "semantic" allows us to find
predominantly Computer Science papers detailing how semantic analysis can help with this cause. Some linguistic journals do
appear with these keywords, but with the current rise of the Large-Language Model, semantic research is seeing a heavy Computer Science edge.

%Databases accessed
For searching these terms, Google Scholar was the primary resource, though GALILEO also was utilized for accessing some non-free files. TODO expand

%and Inclusion and exclusion criteria
And to wrap up our selection process, our primary exclusion criteria was mentioned earlier, and that is a Computer Science focus.
Some articles were industry paid, so were more like Sociology in nature.
And for other articles, they were focusing on the behavioral aspects of phishing and spam only, which while useful, was not in scope for this class.
And speaking of scope, the primary \textit{inclusion} criteria was related to the course's advisements, and that is publication year.
Many useful articles from the beginning of semantic analysis for spam prevention were made in the late 2000's, but they were specifically asked to be ignored.
As such, only 'recent' research was provided, meaning publication dates past 2020 as much as it could be helped. 
These criteria, alongside with the above requirements and toolsets, allowed us to choose the below 8 papers to systematically review.

\section{Literature Review}
%% Summarize and critically analyze each article, focusing on:
%\begin{itemize}
    %\item The problem addressed
    %\item Methods or approaches used
    %\item Key findings and contributions
    %\item Limitations or gaps
%\end{itemize}

\subsection{2016: Overconfidence in Phishing Email Detection} % @j-recasens I am thinking going chronological might work best for easiness

\subsection{20XX: TODO}

\subsection{20XX: TODO}

\subsection{2020: A semantic-based classification approach for an enhanced spam detection}

\subsection{2020: Detecting Phishing SMS Based on Multiple Correlation Algorithms}

\subsection{2022: Examining Factors Impacting the Effectiveness of Anti-Phishing Trainings}

\subsection{20XX: TODO}

\subsection{20XX: TODO}

\section{Synthesis and Conclusion}
%% Highlight overall trends and research gaps, and describe how your project addresses these gaps.

% @j-recasens As an FYI, I've learned that even if a source is in 
% litreview.bib, it won't show up in the compiled code without being \cite'd first.
\bibliographystyle{plain} 
\bibliography{litreview}

\end{document}




%%%% Help Messages from IEEE:

%% \subsection{Identify the Headings}

% Headings, or heads, are organizational devices that guide the reader through 
% your paper. There are two types: component heads and text heads.

% Component heads identify the different components of your paper and are not 
% topically subordinate to each other. Examples include Acknowledgments and 
% References and, for these, the correct style to use is ``Heading 5''. Use 
% ``figure caption'' for your Figure captions, and ``table head'' for your 
% table title. Run-in heads, such as ``Abstract'', will require you to apply a 
% style (in this case, italic) in addition to the style provided by the drop 
% down menu to differentiate the head from the text.

% Text heads organize the topics on a relational, hierarchical basis. For 
% example, the paper title is the primary text head because all subsequent 
% material relates and elaborates on this one topic. If there are two or more 
% sub-topics, the next level head (uppercase Roman numerals) should be used 
% and, conversely, if there are not at least two sub-topics, then no subheads 
% should be introduced.